% !TEX encoding = UTF-8 Unicode

\documentclass[a4paper]{article}

\usepackage{color}
\usepackage{url}
\usepackage[T2A]{fontenc} % enable Cyrillic fonts
\usepackage[utf8]{inputenc} % make weird characters work
\usepackage{graphicx}

\usepackage[english,serbian]{babel}
%\usepackage[english,serbianc]{babel} %ukljuciti babel sa ovim opcijama, umesto gornjim, ukoliko se koristi cirilica

\usepackage[unicode]{hyperref}
\hypersetup{colorlinks,citecolor=green,filecolor=green,linkcolor=blue,urlcolor=blue}

%\newtheorem{primer}{Пример}[section] %ćirilični primer
\newtheorem{primer}{Primer}[section] %biće redefinisano kasnije


\begin{document}

\title{Vrste beskonačnosti. Paradoks Hilbertovog hotela\\ \small{Seminarski rad u okviru kursa\\Tehničko i naučno pisanje\\ Matematički fakultet}}

\author{Milica Zubljić\\ Branko Katanić\\ Dimitrije Spasojević\\ Luka Tonić\\ mi22047@alas.matf.bg.ac.rs}
\date{24.~oktobar 2017.} %biće ažuriran pre predaje rada
\maketitle

\abstract{
U ovom seminarskom radu biće govora o pojmu beskonačnosti. Objasnićemo šta je to beskonačnost, da li postoji više vrsta beskonačnosti, kako se interpretira beskonačnost u svakodnevnom životu itd. Ovaj interesantan koncept, kao i njegova (ne)pojmljivost ljudskom umu, biće ilustrovani kroz primer čuvenog paradoksa Hilbertovog (beskonačnog) hotela.\\

ključne reči: beskonačnost, prebrojivost, neprebrojivost
}

\tableofcontents %sadržaj

\newpage

\section{Beskonačan broj gostiju}
\label{sec:beskonačan broj gostiju}
Pred hotelom se sada nalazi beskonačno veliki autobus sa \textbf {prebrojivo} beskonačno mnogo putnika.
\begin {itemize}
\item Za neki skup kažemo da je prebrojiv ako postoji bijekcija f sa skupa prirodnih brojeva u taj skup.
Drugim rečima, skup je prebrojiv ako se njegovi članovi mogu poredjati u niz.
Jasno je da je činjenica da je gostiju \textbf {prebrojivo} mnogo ključna ako hocemo da ih rasporedimo po sobama. 
\end {itemize}
Ovaj problem možemo rešiti tako što ćemo gosta iz prve sobe zamoliti da predje u drugu, gosta iz druge u četvrtu, iz treće u šestu i tako dalje.
Uopšteno, gost iz sobe \textbf {n} prelazi u sobu \textbf {2n}. Ovim procesom smo oslobili sve sobe sa neparnim brojem.
Pošto neparnih brojeva ima beskonačno mnogo, to ostavlja dovoljno mesta za sve putnike beskonačno velikog autobusa.

\section{Uvod}
\label{poglavlje:uvod}



\section{Kako smestiti beskonačno ljudi iz beskonačno autbusa u hotel}
Jedne noći, dešava se nezamislivo. Noćni poslovođa gleda napolje i vidi beskonačnu kolonu beskonačno velikih autobusa, svaki sa prebrojivo beskonačnim brojem putnika. Ako ne nađe sobe za njih hotel će izgubiti beskonačnu sumu novca, a on će sigurno izgubiti posao. Srećom, seti se da je oko 300. godine p.n.e Euklid dokazao da postoji beskonačan broj prostih brojeva. Da bi ispunio ovaj naizgled nemoguć zadatak da nađe beskonačan broj kreveta za beskonačne autobuse, pune umornih putnika u beskonačnom broju, noćni poslovođa dodeljuje svakom već postojećem gostu prvi prost broj, 2, stepenovan brojem njihove sobe. Tako gost iz sobe broj 7 odlazi u sobu broj $2^7$, a to je soba 128. Poslovođa zatim uzima ljude iz prvog od beskonačnih autobusa i dodeljuje im sobu broj: sledeći prost broj, 3, stepenovan brojem njihovog sedišta u autobusu. Tako osoba na sedištu broj 7 u prvom autobusu odlazi u sobu broj $3^7$ to jest u sobu 2.187. To se nastavlja za ceo prvi autobus. Putnicima iz drugog autobusa se dodeljuju eksponenti sledećeg prostog broja, 5. Sledećem autobusu, eksponenti broja 7. Sledećim autobusima: eksponenti broja 11, eksponenti broja 13, broja 17 i tako dalje.

\subsection{Euklidov dokaz, Beskonačno prostih brojeva}
Neka je dat bilo koji konačni skup prirodnih brojeva p1, p2, ..., pn. Biće pokazano da postoji barem još jedan prost broj koji se ne nalazi u listi. Neka je P proizvod svih prostih brojeva u skupu: P = p1p2...pn. Neka je q = P + 1. Tada q ili jeste ili nije prost broj:
\begin{itemize}
\item Ako je q prost, onda postoji bar jedan broj (q) koji je prost, a nije u prvobitnom skupu.
\item Ako q nije prost, onda neki prost broj p deli q. Kad bi ovaj broj p bio u skupu, tada bi on delio P (jer je P proizvod svih brojeva u skupu); ali p deli P + 1 = q. Ako p deli P i q, onda bi p morao da deli razliku
\cite{objasnjenje_deljenja} ova dva broja, što je (P + 1) − P ili jednostavno 1. Kako nijedan prost broj ne deli 1, p ne može biti u skupu. Ovo znači da barem još jedan prost broj mora postojati mimo onih koji su već u skupu.
\end{itemize} 

Ovo dokazuje da za svaki konačni skup prostih brojeva postoji prost broj koji nije u tom skupu, i stoga prostih brojeva mora biti beskonačno mnogo.


\addcontentsline{toc}{section}{Literatura}
\appendix

\iffalse
\bibliography{seminarski} 
\bibliographystyle{plain}
\fi

\begin{thebibliography}{9}

\bibitem{objasnjenje_deljenja}U opštem slučaju, za svaka tri cela broja a, b, c ako $a \mid b$ i $a \mid c$ onda $a \mid (b - c)$.


\end{thebibliography}


\appendix
\section{Dodatak}
Ovde pišem dodatne stvari, ukoliko za time ima potrebe.
Ovde pišem dodatne stvari, ukoliko za time ima potrebe.
Ovde pišem dodatne stvari, ukoliko za time ima potrebe.
Ovde pišem dodatne stvari, ukoliko za time ima potrebe.
Ovde pišem dodatne stvari, ukoliko za time ima potrebe.


\end{document}
