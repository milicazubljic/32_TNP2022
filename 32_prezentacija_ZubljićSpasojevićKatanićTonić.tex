\documentclass{beamer}
\usepackage{beamerthemeshadow}
\usepackage{graphicx}
\usepackage{color}
\usepackage[utf8]{inputenc}
\usepackage{hyperref}
\definecolor{darkblue}{rgb}{0.11,0.27,0.55}
\setbeamercolor{structure}{fg=darkblue}

\def\d{{\fontencoding{T1}\selectfont\dj}}
\def\D{{\fontencoding{T1}\selectfont\DJ}}

\title{Vrste beskonačnosti. Paradoks Hilbertovog hotela}
\subtitle{-- Tehničko i naučno pisanje --}
\author{Milica Zubljić, Dimitrije Spasojević, Branko Katanić, Luka Tonić}
\institute{Matematički fakultet\\Univerzitet u Beogradu}
\date{
	\footnotesize{Beograd, 2022.}	
}

\begin{document}
\begin{frame}
	\thispagestyle{empty}
	\titlepage
\end{frame}

\addtocounter{framenumber}{-1}

\begin{frame}
	\frametitle{Sadržaj} 
	\tableofcontents[hidesubsections] 
\end{frame}

\section{Hilbertov hotel}
\begin{frame}[fragile]\frametitle{Konačan broj gostiju}
\begin{itemize}
    \item \textbf{Hilbertov hotel}-Hotel sa beskonačno mnogo soba
    \item Ako je ovaj hotel pun, da li postoji način da oslobodimo još jednu sobu?
    \begin{itemize}
        \item Kada svakog gosta iz sobe $n$ zamolimo da predje u sobu $n+1$, prva soba će ostati prazna
        \item Ista formula važi za svaki konačan broj gostiju
    \end{itemize}
\end{itemize}
\end{frame}

\begin{frame}[fragile]\frametitle{Beskonačan broj gostiju}
\begin{itemize}
    \item Pred hotelom je sada beskonačno veliki autobus sa prebrojivo beskonačno mnogo gostiju
    \item Problem postaje nešto komplikovaniji kada treba napraviti mesta za beskonačno novih gostiju
    \item Sada će svaki gost iz sobe $n$ preći u sobu $2n$, i tako ćemo osloboditi sve sobe sa neparnim brojem
    \item Pošto neparnih brojeva ima beskonačno mnogo, ovim procesom oslobodili smo beskonačno mnogo soba
\end{itemize}
\end{frame}

\end {document}